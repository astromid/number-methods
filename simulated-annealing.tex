% Default to the notebook output style

% Inherit from the specified cell style.




    
\documentclass{article}

    
    

    \usepackage[T1]{fontenc}
    % Nicer default font (+ math font) than Computer Modern for most use cases
    \usepackage{mathpazo}

    % Basic figure setup, for now with no caption control since it's done
    % automatically by Pandoc (which extracts ![](path) syntax from Markdown).
    \usepackage{graphicx}
    % We will generate all images so they have a width \maxwidth. This means
    % that they will get their normal width if they fit onto the page, but
    % are scaled down if they would overflow the margins.
    \makeatletter
    \def\maxwidth{\ifdim\Gin@nat@width>\linewidth\linewidth
    \else\Gin@nat@width\fi}
    \makeatother
    \let\Oldincludegraphics\includegraphics
    % Set max figure width to be 80% of text width, for now hardcoded.
    \renewcommand{\includegraphics}[1]{\Oldincludegraphics[width=.8\maxwidth]{#1}}
    % Ensure that by default, figures have no caption (until we provide a
    % proper Figure object with a Caption API and a way to capture that
    % in the conversion process - todo).
    \usepackage{caption}
    \DeclareCaptionLabelFormat{nolabel}{}
    \captionsetup{labelformat=nolabel}

    \usepackage{adjustbox} % Used to constrain images to a maximum size 
    \usepackage{xcolor} % Allow colors to be defined
    \usepackage{enumerate} % Needed for markdown enumerations to work
    \usepackage{geometry} % Used to adjust the document margins
    \usepackage{amsmath} % Equations
    \usepackage{amssymb} % Equations
    \usepackage{textcomp} % defines textquotesingle
    % Hack from http://tex.stackexchange.com/a/47451/13684:
    \AtBeginDocument{%
        \def\PYZsq{\textquotesingle}% Upright quotes in Pygmentized code
    }
    \usepackage{upquote} % Upright quotes for verbatim code
    \usepackage{eurosym} % defines \euro
    \usepackage[mathletters]{ucs} % Extended unicode (utf-8) support
    \usepackage[utf8x]{inputenc} % Allow utf-8 characters in the tex document
    \usepackage{fancyvrb} % verbatim replacement that allows latex
    \usepackage{grffile} % extends the file name processing of package graphics 
                         % to support a larger range 
    % The hyperref package gives us a pdf with properly built
    % internal navigation ('pdf bookmarks' for the table of contents,
    % internal cross-reference links, web links for URLs, etc.)
    \usepackage{hyperref}
    \usepackage{longtable} % longtable support required by pandoc >1.10
    \usepackage{booktabs}  % table support for pandoc > 1.12.2
    \usepackage[inline]{enumitem} % IRkernel/repr support (it uses the enumerate* environment)
    \usepackage[normalem]{ulem} % ulem is needed to support strikethroughs (\sout)
                                % normalem makes italics be italics, not underlines
     % load all other packages
\usepackage[12pt]{extsizes}
\usepackage[T2A]{fontenc}
\usepackage{amssymb,amsmath,mathtext}
\usepackage{indentfirst,amsfonts}
\usepackage[english, russian]{babel}
\usepackage{setspace,amsmath}


    
    % Colors for the hyperref package
    \definecolor{urlcolor}{rgb}{0,.145,.698}
    \definecolor{linkcolor}{rgb}{.71,0.21,0.01}
    \definecolor{citecolor}{rgb}{.12,.54,.11}

    % ANSI colors
    \definecolor{ansi-black}{HTML}{3E424D}
    \definecolor{ansi-black-intense}{HTML}{282C36}
    \definecolor{ansi-red}{HTML}{E75C58}
    \definecolor{ansi-red-intense}{HTML}{B22B31}
    \definecolor{ansi-green}{HTML}{00A250}
    \definecolor{ansi-green-intense}{HTML}{007427}
    \definecolor{ansi-yellow}{HTML}{DDB62B}
    \definecolor{ansi-yellow-intense}{HTML}{B27D12}
    \definecolor{ansi-blue}{HTML}{208FFB}
    \definecolor{ansi-blue-intense}{HTML}{0065CA}
    \definecolor{ansi-magenta}{HTML}{D160C4}
    \definecolor{ansi-magenta-intense}{HTML}{A03196}
    \definecolor{ansi-cyan}{HTML}{60C6C8}
    \definecolor{ansi-cyan-intense}{HTML}{258F8F}
    \definecolor{ansi-white}{HTML}{C5C1B4}
    \definecolor{ansi-white-intense}{HTML}{A1A6B2}

    % commands and environments needed by pandoc snippets
    % extracted from the output of `pandoc -s`
    \providecommand{\tightlist}{%
      \setlength{\itemsep}{0pt}\setlength{\parskip}{0pt}}
    \DefineVerbatimEnvironment{Highlighting}{Verbatim}{commandchars=\\\{\}}
    % Add ',fontsize=\small' for more characters per line
    \newenvironment{Shaded}{}{}
    \newcommand{\KeywordTok}[1]{\textcolor[rgb]{0.00,0.44,0.13}{\textbf{{#1}}}}
    \newcommand{\DataTypeTok}[1]{\textcolor[rgb]{0.56,0.13,0.00}{{#1}}}
    \newcommand{\DecValTok}[1]{\textcolor[rgb]{0.25,0.63,0.44}{{#1}}}
    \newcommand{\BaseNTok}[1]{\textcolor[rgb]{0.25,0.63,0.44}{{#1}}}
    \newcommand{\FloatTok}[1]{\textcolor[rgb]{0.25,0.63,0.44}{{#1}}}
    \newcommand{\CharTok}[1]{\textcolor[rgb]{0.25,0.44,0.63}{{#1}}}
    \newcommand{\StringTok}[1]{\textcolor[rgb]{0.25,0.44,0.63}{{#1}}}
    \newcommand{\CommentTok}[1]{\textcolor[rgb]{0.38,0.63,0.69}{\textit{{#1}}}}
    \newcommand{\OtherTok}[1]{\textcolor[rgb]{0.00,0.44,0.13}{{#1}}}
    \newcommand{\AlertTok}[1]{\textcolor[rgb]{1.00,0.00,0.00}{\textbf{{#1}}}}
    \newcommand{\FunctionTok}[1]{\textcolor[rgb]{0.02,0.16,0.49}{{#1}}}
    \newcommand{\RegionMarkerTok}[1]{{#1}}
    \newcommand{\ErrorTok}[1]{\textcolor[rgb]{1.00,0.00,0.00}{\textbf{{#1}}}}
    \newcommand{\NormalTok}[1]{{#1}}
    
    % Additional commands for more recent versions of Pandoc
    \newcommand{\ConstantTok}[1]{\textcolor[rgb]{0.53,0.00,0.00}{{#1}}}
    \newcommand{\SpecialCharTok}[1]{\textcolor[rgb]{0.25,0.44,0.63}{{#1}}}
    \newcommand{\VerbatimStringTok}[1]{\textcolor[rgb]{0.25,0.44,0.63}{{#1}}}
    \newcommand{\SpecialStringTok}[1]{\textcolor[rgb]{0.73,0.40,0.53}{{#1}}}
    \newcommand{\ImportTok}[1]{{#1}}
    \newcommand{\DocumentationTok}[1]{\textcolor[rgb]{0.73,0.13,0.13}{\textit{{#1}}}}
    \newcommand{\AnnotationTok}[1]{\textcolor[rgb]{0.38,0.63,0.69}{\textbf{\textit{{#1}}}}}
    \newcommand{\CommentVarTok}[1]{\textcolor[rgb]{0.38,0.63,0.69}{\textbf{\textit{{#1}}}}}
    \newcommand{\VariableTok}[1]{\textcolor[rgb]{0.10,0.09,0.49}{{#1}}}
    \newcommand{\ControlFlowTok}[1]{\textcolor[rgb]{0.00,0.44,0.13}{\textbf{{#1}}}}
    \newcommand{\OperatorTok}[1]{\textcolor[rgb]{0.40,0.40,0.40}{{#1}}}
    \newcommand{\BuiltInTok}[1]{{#1}}
    \newcommand{\ExtensionTok}[1]{{#1}}
    \newcommand{\PreprocessorTok}[1]{\textcolor[rgb]{0.74,0.48,0.00}{{#1}}}
    \newcommand{\AttributeTok}[1]{\textcolor[rgb]{0.49,0.56,0.16}{{#1}}}
    \newcommand{\InformationTok}[1]{\textcolor[rgb]{0.38,0.63,0.69}{\textbf{\textit{{#1}}}}}
    \newcommand{\WarningTok}[1]{\textcolor[rgb]{0.38,0.63,0.69}{\textbf{\textit{{#1}}}}}
    
    
    % Define a nice break command that doesn't care if a line doesn't already
    % exist.
    \def\br{\hspace*{\fill} \\* }
    % Math Jax compatability definitions
    \def\gt{>}
    \def\lt{<}
    % Document parameters
    \title{simulated-annealing}
    
    
    

    
    % Prevent overflowing lines due to hard-to-break entities
    \sloppy 
    % Setup hyperref package
    \hypersetup{
      breaklinks=true,  % so long urls are correctly broken across lines
      colorlinks=true,
      urlcolor=urlcolor,
      linkcolor=linkcolor,
      citecolor=citecolor,
      }
    % Slightly bigger margins than the latex defaults
    
    \geometry{verbose,tmargin=1in,bmargin=1in,lmargin=1in,rmargin=1in}
    
    

    \begin{document}
    
    
    
    
    

    
    \section{Методы стохастической
оптимизации}\label{ux43cux435ux442ux43eux434ux44b-ux441ux442ux43eux445ux430ux441ux442ux438ux447ux435ux441ux43aux43eux439-ux43eux43fux442ux438ux43cux438ux437ux430ux446ux438ux438}

    \subsection{Метод имитации отжига (simulated
annealing)}\label{ux43cux435ux442ux43eux434-ux438ux43cux438ux442ux430ux446ux438ux438-ux43eux442ux436ux438ux433ux430-simulated-annealing}

    \subsubsection{Описание}\label{ux43eux43fux438ux441ux430ux43dux438ux435}

    Метод имитации отжига - это метод стохастической оптимизации,
использующий упорядоченный случайный поиск на основе аналогии с
процессом образования в веществе кристаллической структуры при
охлаждении. Преимуществом метода отжига являются возможность избегать
локальных минимумов оптимизируемой функции за счет принятия временно
ухудшающих результат решений, что отражает суть процесса нагрева
расплава для предотвращения его быстрого остывания. Метод отжига
отличается от итеративных алгоритмов адаптивностью.

    Метод отжига является одним из алгоритмов поиска глобального экстремума
целевой функции \(f(x)\), заданной для \(x \in S\). Элементы множества
\(S\), дискретного или непрерывного, представляют собой состояния
воображаемой физической системы("энергетические уровни"). Значение
функции \(f\) в этих точках используется как энергия системы
\(E = f(x)\). В каждый момент времени температура системы предполагается
заданной. Находясь в состоянии с температурой \(T\), следующее состояние
системы выбирается в соответствии с заданным семейством вероятностных
распределений \(Q(x, T)\), которое задает новый случайный элемент
\(x^l = G(x, T)\). После генерации \(x^l\) система с вероятностью
\(h(\Delta E; T)\) переходит к следующему шагу в следующее состояние
\(x^l\). Если этот переход не произошел, процесс генерации \(x^l\)
повторяется. Здесь \(\Delta E\) обозначает приращение функции
\(f(x^l) - f(l)\)

    Конкретная схема отжига задается следующими параметрами: * Выбором
закона изменения температуры \(T(k)\) на шаге \(k\) * Выбором
вероятностного распределения \(Q(x; T)\) * Выбором функции вероятности
принятия решения \(h(\Delta E; T)\)

    В общем виде алгоритм можно представить в следующем виде: 1. Случайным
образом выбирается начальная точка \(x = x_0; x_0 \in S\). Текущее
значение энергии \(E\) устанавливается равным \(f(x_0)\). 2. \(k\)-я
итерация основного цикла состоит из шагов: * Сравнить энергию системы
\(E\) в состоянии \(x\) с найденным на данный момент глобальным
минимумом. Если \(E = f(x)\) меньше, то изменить значение глобального
минимума. * Сгенерировать новую точку \(x^l = G(x; T(k))\). * Вычислить
в ней значение \(E^l = f(x^l)\). * Сгенерировать случайное число
\(\alpha \sim U[0;1]\) * Если \(\alpha < h(E^l - E; T(k))\), то: * *
установить \(x \longrightarrow x^l; E \longrightarrow E^l\) * * Перейти
к следующей итерации * Иначе, повторить процесс генерации

    \subsubsection{Реализация
(Python)}\label{ux440ux435ux430ux43bux438ux437ux430ux446ux438ux44f-python}

    \subsubsection{Исследование}\label{ux438ux441ux441ux43bux435ux434ux43eux432ux430ux43dux438ux435}

    Проверим сначала реализацию на каком-нибудь простом примере, например,
квадратичной функции \(f(x) = 5x^2 + 6x - 4\):

    
    
    \begin{verbatim}
[<matplotlib.lines.Line2D at 0x7f0747a53080>]
    \end{verbatim}

    

    \begin{center}
    \adjustimage{max size={0.9\linewidth}{0.9\paperheight}}{output_16_1.png}
    \end{center}
    { \hspace*{\fill} \\}
    
    \begin{Verbatim}[commandchars=\\\{\}]
Minimum found: state = -0.5957411989856904, E = -5.799909313069603

    \end{Verbatim}

    Аналитический минимум: \(x_{min} = -\frac{3}{5}\), \$f(x\_\{min\}) =
-\frac{29}{5} \$, что с хорошей точностью совпадает с полученным отжигом
результатом.

    Исследуем поведение отжига на более сложной функции:

    
    
    \begin{verbatim}
[<matplotlib.lines.Line2D at 0x7f074753da20>]
    \end{verbatim}

    

    \begin{center}
    \adjustimage{max size={0.9\linewidth}{0.9\paperheight}}{output_20_1.png}
    \end{center}
    { \hspace*{\fill} \\}
    
    Переберем несколько наборов параметров: * \(T_0 = [100, 10, 1]\) *
\(T_f = [0.01, 0.001, 0.0001, 0.00001]\) * \(\alpha = [1, 0.1, 0.01]\)

    Отобразим полученные результаты на графике функции:

    
    
    \begin{verbatim}
[<matplotlib.lines.Line2D at 0x7f0746fdf400>]
    \end{verbatim}

    

    \begin{center}
    \adjustimage{max size={0.9\linewidth}{0.9\paperheight}}{output_25_1.png}
    \end{center}
    { \hspace*{\fill} \\}
    
    Посмотрим на отдельные зависимости:

    
    
    \begin{verbatim}
[<matplotlib.lines.Line2D at 0x7f0746c7acf8>]
    \end{verbatim}

    

    \begin{center}
    \adjustimage{max size={0.9\linewidth}{0.9\paperheight}}{output_27_1.png}
    \end{center}
    { \hspace*{\fill} \\}
    
    Видно, что качество найденного решения сильно зависит от конечной
температуры, и слабо - от начальной и коэффициента альфа(они влияют на
время выполнения, т.к. от них зависит итоговое количество итераций).


    % Add a bibliography block to the postdoc
    
    
    
    \end{document}
